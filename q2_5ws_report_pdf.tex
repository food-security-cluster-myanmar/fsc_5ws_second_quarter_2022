% Options for packages loaded elsewhere
\PassOptionsToPackage{unicode}{hyperref}
\PassOptionsToPackage{hyphens}{url}
\PassOptionsToPackage{dvipsnames,svgnames,x11names}{xcolor}
%
\documentclass[
]{article}
\usepackage{amsmath,amssymb}
\usepackage{lmodern}
\usepackage{iftex}
\ifPDFTeX
  \usepackage[T1]{fontenc}
  \usepackage[utf8]{inputenc}
  \usepackage{textcomp} % provide euro and other symbols
\else % if luatex or xetex
  \usepackage{unicode-math}
  \defaultfontfeatures{Scale=MatchLowercase}
  \defaultfontfeatures[\rmfamily]{Ligatures=TeX,Scale=1}
\fi
% Use upquote if available, for straight quotes in verbatim environments
\IfFileExists{upquote.sty}{\usepackage{upquote}}{}
\IfFileExists{microtype.sty}{% use microtype if available
  \usepackage[]{microtype}
  \UseMicrotypeSet[protrusion]{basicmath} % disable protrusion for tt fonts
}{}
\makeatletter
\@ifundefined{KOMAClassName}{% if non-KOMA class
  \IfFileExists{parskip.sty}{%
    \usepackage{parskip}
  }{% else
    \setlength{\parindent}{0pt}
    \setlength{\parskip}{6pt plus 2pt minus 1pt}}
}{% if KOMA class
  \KOMAoptions{parskip=half}}
\makeatother
\usepackage{xcolor}
\usepackage[margin=1in]{geometry}
\usepackage{graphicx}
\makeatletter
\def\maxwidth{\ifdim\Gin@nat@width>\linewidth\linewidth\else\Gin@nat@width\fi}
\def\maxheight{\ifdim\Gin@nat@height>\textheight\textheight\else\Gin@nat@height\fi}
\makeatother
% Scale images if necessary, so that they will not overflow the page
% margins by default, and it is still possible to overwrite the defaults
% using explicit options in \includegraphics[width, height, ...]{}
\setkeys{Gin}{width=\maxwidth,height=\maxheight,keepaspectratio}
% Set default figure placement to htbp
\makeatletter
\def\fps@figure{htbp}
\makeatother
\setlength{\emergencystretch}{3em} % prevent overfull lines
\providecommand{\tightlist}{%
  \setlength{\itemsep}{0pt}\setlength{\parskip}{0pt}}
\setcounter{secnumdepth}{-\maxdimen} % remove section numbering
\usepackage{float}
\usepackage{booktabs}
\usepackage{longtable}
\usepackage{array}
\usepackage{multirow}
\usepackage{wrapfig}
\usepackage{float}
\usepackage{colortbl}
\usepackage{pdflscape}
\usepackage{tabu}
\usepackage{threeparttable}
\usepackage{threeparttablex}
\usepackage[normalem]{ulem}
\usepackage{makecell}
\usepackage{xcolor}
\ifLuaTeX
  \usepackage{selnolig}  % disable illegal ligatures
\fi
\IfFileExists{bookmark.sty}{\usepackage{bookmark}}{\usepackage{hyperref}}
\IfFileExists{xurl.sty}{\usepackage{xurl}}{} % add URL line breaks if available
\urlstyle{same} % disable monospaced font for URLs
\hypersetup{
  pdftitle={Report on 5Ws: Mid-Year 2022},
  pdfauthor={Myanmar Food Security Cluster},
  colorlinks=true,
  linkcolor={red},
  filecolor={Maroon},
  citecolor={Blue},
  urlcolor={blue},
  pdfcreator={LaTeX via pandoc}}

\title{\textbf{Report on 5Ws: Mid-Year 2022}}
\author{Myanmar Food Security Cluster}
\date{2022/08/01}

\begin{document}
\maketitle

{
\hypersetup{linkcolor=}
\setcounter{tocdepth}{4}
\tableofcontents
}
\href{https://food-security-cluster-myanmar.github.io/}{Food Security
Cluster Myanmar homepage}

\hypertarget{summary-of-achievements}{%
\subsection{Summary of achievements}\label{summary-of-achievements}}

Beneficiaries of humanitarian action formed 90.57\% of the 3,033,156
beneficiaries in the first quarter of 2022. The remainder were reached
through development interventions.

To recall, the Food Security Cluster's strategic objectives for 2022
are:

\begin{itemize}
\tightlist
\item
  SO1: IDPs have equitable access to sufficient, safe and nutritious
  food (either in-kind or through food assistance)
\item
  SO2: Vulnerable persons (excl. IDPs) have equitable access to
  sufficient, safe and nutritious food (either in-kind or through food
  assistance)
\item
  SO3: Restore, protect and improve livelihoods and resilience
\end{itemize}

Overall, 91.12\% of the food security cluster's beneficiaries were from
humanitarian activities.

2022 Q1 \& Q2 humanitarian beneficiaries

SO

Q1

Q2

Total

total\%

SO1

193,154

341,050

534,204

19.33

SO2

1,560,453

588,488

2,148,941

77.75

SO3

33,116

47,686

80,802

2.92

Total

1,786,723

977,224

2,763,947

100.00

A total of 8.88\% beneficiaries were from development activities and
actors.

2022 Q1 \& Q2 development beneficiaries

SO

Q1

Q2

Total

total\%

SO2

1,000

16

1,016

0.38

SO3

220,587

47,606

268,193

99.62

Total

221,587

47,622

269,209

100.00

Activities with significant increases in beneficiaries reached compared
to Q1 include food/cash for work/assets, multi-purpose cash transfers
and vocational training. New persons reached by farmer field school and
farmer training dipped, likely due to the seasonality of the
intervention. Encouragingly, the number of new persons reached by food
distributions in Q2 was 83.08\% as opposed to 85.81\% in Q1, showing
increasing investment in more durable solutions. Myanmar will likely
face a prolonged food security crisis.

\begin{table}

\caption{\label{tab:table-activity}Beneficiaries by activity, Q1 & Q2 2022}
\centering
\begin{tabular}[t]{l|r|r|r|r}
\hline
Activity & Q1 & Q2 & \%\_change & Total\\
\hline
Food distribution & 1,723,250 & 851,483 & -50.59 & 2,574,733\\
\hline
FFS and farmer training & 195,839 & 1,003 & -99.49 & 196,842\\
\hline
Crop, vegetable and seed kits & 48,046 & 60,912 & 26.78 & 108,958\\
\hline
Multi-purpose cash transfer & 31,357 & 76,365 & 143.53 & 107,722\\
\hline
Food\_cash for work\_assets & 7,352 & 26,362 & 258.57 & 33,714\\
\hline
Community infrastructure and equipment & 0 & 4,770 & 0.00 & 4,770\\
\hline
IGA and small grants & 2,048 & 462 & -77.44 & 2,510\\
\hline
Heb and fortfied rice & 0 & 1,706 & 0.00 & 1,706\\
\hline
Vocational training & 327 & 1,306 & 299.39 & 1,633\\
\hline
Kitchen garden kits & 0 & 475 & 0.00 & 475\\
\hline
Livestock kits & 91 & 2 & -97.80 & 93\\
\hline
\end{tabular}
\end{table}

Food distributions overwhelmingly target moderately-food-insecure
persons in host and local communities, this group forms 75.04\% of all
beneficiaries of food distributions.

\begin{table}

\caption{\label{tab:food-distributions-fs-status}Food distributions by type food insecurity status and beneficiary type}
\centering
\begin{tabular}[t]{l|r|r|r}
\hline
beneficiary\_type & Moderate & Severe & Total\\
\hline
Host/local Community & 1,913,480 & 18,655 & 1,932,135\\
\hline
Internally Displaced & 260,257 & 182,793 & 443,050\\
\hline
Rakhine Stateless & 3,618 & 189,405 & 193,023\\
\hline
Resettled & 2,977 & 0 & 2,977\\
\hline
Returnees & 2,413 & 1,135 & 3,548\\
\hline
Total & 2,182,745 & 391,988 & 2,574,733\\
\hline
\end{tabular}
\end{table}

55\% of beneficiaries were reached by activities where nutrition had
been mainstreamed.

\begin{table}

\caption{\label{tab:table-nutrition-mainstreaming}Beneficiaries by status of nutrition mainstreaming}
\centering
\begin{tabular}[t]{l|r|r|r|r|r}
\hline
was\_nutrition\_mainstreamed\_in\_activity & SO1 & SO2 & SO3 & total\_beneficiaries & \%\_beneficiaries\\
\hline
Yes & 327,046 & 957,282 & 90,490 & 1,374,818 & 45.33\\
\hline
No & 207,158 & 1,192,675 & 258,505 & 1,658,338 & 54.67\\
\hline
\end{tabular}
\end{table}

\hypertarget{geographies}{%
\subsection{1. Geographies}\label{geographies}}

\hypertarget{states}{%
\subsubsection{1.1 States}\label{states}}

Though new beneficiaries reached remained biased towards Yangon and
Rakhine in Q2, figures were less skewed than they were in Q1. Overall
66.24\% of beneficiaries in Q2 came from Yangon or Rakhine, whereas it
was 78.83\% in Q1. Kayah saw the largest quarter-to-quarter increase in
number of beneficiaries.

\includegraphics{q2_5ws_report_pdf_files/figure-latex/facet-state-quarter-1.pdf}

A total of 123 townships have been reached across 16 states/regions.

\hypertarget{townships}{%
\subsubsection{1.2 Townships}\label{townships}}

The top 7 townships, each from Yangon or Rakhine contained 68\% of all
beneficiaries.

\begin{table}

\caption{\label{tab:table-top-townships}Top townships by beneficiaries reached}
\centering
\begin{tabular}[t]{l|l|r|r|r|r}
\hline
state & township & Q1 & Q2 & Total & \%total\\
\hline
Yangon & Hlaingtharya (West) & 433,074 & 36,195 & 469,269 & 15.47\\
\hline
Yangon & Hlaingtharya (East) & 270,646 & 128,461 & 399,107 & 13.16\\
\hline
Yangon & Shwepyithar & 380,512 & 0 & 380,512 & 12.55\\
\hline
Yangon & North Okkalapa & 168,400 & 172,559 & 340,959 & 11.24\\
\hline
Rakhine & Buthidaung & 147,534 & 18,314 & 165,848 & 5.47\\
\hline
Yangon & Dala & 81,125 & 81,390 & 162,515 & 5.36\\
\hline
Rakhine & Sittwe & 22,484 & 130,281 & 152,765 & 5.04\\
\hline
Kayah & Loikaw & 815 & 107,473 & 108,288 & 3.57\\
\hline
Mandalay & Nyaung-U & 71,547 & 0 & 71,547 & 2.36\\
\hline
Rakhine & Maungdaw & 44,625 & 22,757 & 67,382 & 2.22\\
\hline
Mandalay & Myingyan & 46,087 & 3 & 46,090 & 1.52\\
\hline
Kayin & Hpapun & 12,477 & 29,561 & 42,038 & 1.39\\
\hline
Kayah & Hpruso & 26,507 & 7,557 & 34,064 & 1.12\\
\hline
Kachin & Waingmaw & 19,249 & 14,292 & 33,541 & 1.11\\
\hline
Rakhine & Pauktaw & 76 & 31,597 & 31,673 & 1.04\\
\hline
\multicolumn{6}{l}{\rule{0pt}{1em}Only showing townships with >1\% of total beneficiaries}\\
\end{tabular}
\end{table}

Comparing the food security cluster's footprint in the first quarter,
with that from the second quarter, new activity can be noted in
Mandalay, Magway, Kayah and Kayin.

\includegraphics{q2_5ws_report_pdf_files/figure-latex/maps-ben-quarter-1.pdf}

26 new townships were added in the second quarter of 2022, including 5
from Ayeyarwady and 3 each from Mandalay, Magway and Tanintharyi.

\hypertarget{locations}{%
\subsubsection{1.3 Locations}\label{locations}}

A location refers to either a village, ward, IDP site or industrial
zone.

The vast amount of project locations have only one food security
activity.

\includegraphics{q2_5ws_report_pdf_files/figure-latex/unnamed-chunk-1-1.pdf}

With some exceptions, the vast majority of project locations have only
one partner present.

\includegraphics{q2_5ws_report_pdf_files/figure-latex/histogram-parnters-location-1.pdf}

224 villages have more than one partner present. The table below breaks
down the beneficiaries from these locations by activity and state.

\begin{table}

\caption{\label{tab:table-more-than-one-partner}Activities conducted in villages with more than one partner}
\centering
\begin{tabular}[t]{l|l|r}
\hline
state & activity\_red & beneficiaries\\
\hline
Rakhine & food distribution & 287,278\\
\hline
Yangon & food distribution & 65,721\\
\hline
Magway & FFS and farmer training & 2,735\\
\hline
Rakhine & crop, vegetable and seed kits & 1,525\\
\hline
Kachin & crop, vegetable and seed kits & 698\\
\hline
Rakhine & food\_cash for work\_assets & 653\\
\hline
Shan (South) & vocational training & 504\\
\hline
Mon & crop, vegetable and seed kits & 365\\
\hline
Rakhine & vocational training & 66\\
\hline
\end{tabular}
\end{table}

The food security cluster's partners can mostly be found in Yangon,
Rakhine and Kachin.

\includegraphics{q2_5ws_report_pdf_files/figure-latex/locations-partners-state-1.pdf}

\hypertarget{activities}{%
\subsection{2. Activities}\label{activities}}

\hypertarget{progress-by-activity}{%
\subsubsection{2.1 Progress by activity}\label{progress-by-activity}}

The first dotted red line shows the end of Q1 and the second shows the
end of Q2. The thick line in grey shows the progress in 2021 for the
same activity. It is observed that food distributions in 2022 greatly
outpaced those in 2021. The same is also true for cash/work for
food/assets and crop, vegetable and seed kits.

\includegraphics{q2_5ws_report_pdf_files/figure-latex/progress-facet-lineplot-1.pdf}

Newly implemented in Q2 of 2022 was the provision of HEB and fortified
rice, largely in Chin state. Food distributions continued to be the
largest activity, followed by the provision of crop, vegetable and seed
kits.

\hypertarget{agricultural-and-livelihoods-activities}{%
\subsubsection{2.2 Agricultural and livelihoods
activities}\label{agricultural-and-livelihoods-activities}}

The percentage of beneficiaries reached by agriculture and livelihoods
activities (crops, vegetable and seed kits; FFS and farmer training; IGA
and small grants; livestock kits) slightly in Q2 2022, compared to Q1.

\includegraphics{q2_5ws_report_pdf_files/figure-latex/table-agricultural-activity-1.pdf}

The vast majority of beneficiaries of agricultural and livelihoods
activities are recipients of farmer training and crop, vegetable and
seed kits. It is possible that not all of Q2 was available for farmer
training.

\includegraphics{q2_5ws_report_pdf_files/figure-latex/barplot-facet-ag-activities-1.pdf}

\hypertarget{delivery-modalities}{%
\subsubsection{2.3 Delivery modalities}\label{delivery-modalities}}

Only community infrastructure and equipment, HEB and fortified rice and
kitchen garden kits were delivered entirely through in-kind modalities.

\begin{table}

\caption{\label{tab:unnamed-chunk-2}Percentage of beneficiaries reached by activity and delivery modality}
\centering
\begin{tabular}[t]{l|l|l|l|l|r}
\hline
Activity & In-kind & CBT/CVA & Hybrid & Service delivery & Beneficiaries\\
\hline
Food distribution & 86.1\% & 12.3\% & 1.6\% &  & 2,574,733\\
\hline
FFS and farmer training & 15.0\% & 24.7\% &  & 60.3\% & 196,842\\
\hline
Crop, vegetable and seed kits & 83.3\% & 0.5\% & 16.2\% & 0.0\% & 108,958\\
\hline
Multi-purpose cash transfer &  & 89.3\% & 10.7\% &  & 107,722\\
\hline
Food\_cash for work\_assets & 17.3\% & 82.7\% &  &  & 33,714\\
\hline
Community infrastructure and equipment & 100.0\% &  &  &  & 4,770\\
\hline
IGA and small grants & 10.4\% & 87.4\% & 1.2\% & 1.0\% & 2,510\\
\hline
Heb and fortfied rice & 100.0\% &  &  &  & 1,706\\
\hline
Vocational training & 0.0\% & 23.1\% &  & 76.9\% & 1,633\\
\hline
Kitchen garden kits & 100.0\% &  &  &  & 475\\
\hline
Livestock kits & 97.8\% & 0.0\% & 0.0\% & 2.2\% & 93\\
\hline
\end{tabular}
\end{table}

There are also clear differences between the different beneficiary types
and the delivery modalities employed with them. Beneficiaries from
host/local communities largely received in-kind distributions whilst
those from camps and IDP sites mostly received cash-based interventions.

\includegraphics{q2_5ws_report_pdf_files/figure-latex/facet-ben-type-1.pdf}

Areas with more IDPs, such as Sagaing, Rakhine and Kachin, reach most of
their beneficiaries through cash-based programming.

\includegraphics{q2_5ws_report_pdf_files/figure-latex/delivery-modalities-stacked-bar-1.pdf}

\hypertarget{cash-based-programming}{%
\subsection{3. Cash-based programming}\label{cash-based-programming}}

\hypertarget{usd-per-household}{%
\subsubsection{3.1 USD per household}\label{usd-per-household}}

Compared to Q1, beneficiaries in Q2 are much less likely to have
received cash transfers of less than USD 10 per household. The most
common transfer values were between USD 60 and 70, an increase from the
previous quarter.

\includegraphics{q2_5ws_report_pdf_files/figure-latex/usd-hhd-bin-barplot-1.pdf}

\hypertarget{usd-per-person}{%
\subsubsection{3.3 USD per person}\label{usd-per-person}}

The boxplots above shows the range of cash transfer values (all values
are per person, to facilitate comparability) by activity. The average
for reach activity is marked by the thick line in the middle of each
box. The leftmost and rightmost side of each box indicate the 25th and
75th percentile of transfer values, respectively. The length of each box
is a gauge for how much variation there is in the transfer values of
each activity.

\includegraphics{q2_5ws_report_pdf_files/figure-latex/boxplot-activity-usd-per-person-1.pdf}

Each of the bubbles represents an individual distribution, with their
position along the x-axis showing the USD per person value of the
distribution and the size of each bubble indicating the number of
beneficiaries reached.

Food distributions tended to have the tightest range of values.

\includegraphics{q2_5ws_report_pdf_files/figure-latex/plotly-transfer-value-scatter-1.pdf}

Cash transfer values tended to be higher in Q2 as compared to Q1 largely
due to increases in the per-household package of multi-purpose cash
transfers.

\hypertarget{food-distributions}{%
\subsubsection{3.3 Food distributions}\label{food-distributions}}

\includegraphics{q2_5ws_report_pdf_files/figure-latex/plotly-food-dist-range-1.pdf}

Kachin, Rakhine and Shan notably have several extreme outliers much
higher than the average for that state. Kayin, however, has a very large
number of beneficiaries who received less the USD 1/person.
Distributions in Chin and Ayeyarwady had very consistent values as they
were all implemented by the same implementing partner.

The table below compares the different bins for cash transfer values of
food distributions with the minimum expenditure basket for food
established by the Cash Working Group. They have established a floor of
MMK 190,555 (or USD 114.55).

Overall, 1.86\% of food distribution beneficiaries have received at
least 100\% of the MEB and 9.35\% have received at least 50\% of the MEB
(USD 11.45 per person).

\begin{table}

\caption{\label{tab:table-meb-usd-hhd-bin}USD values of food distributions by percentage of MEB received}
\centering
\begin{tabular}[t]{l|r|r|r|>{}r}
\hline
usd\_person\_bin & avg\_pc\_of\_meb & avg\_usd\_month & beneficiaries & pc\_of\_hhd\\
\hline
<\$2 & 5.39 & 1.23 & 20,923 & \cellcolor[HTML]{76D054}{\textcolor{white}{4.76}}\\
\hline
>=\$2\_<\$4 & 14.44 & 3.31 & 49,617 & \cellcolor[HTML]{2AB07F}{\textcolor{white}{11.30}}\\
\hline
>=\$4\_<\$6 & 21.38 & 4.90 & 30,454 & \cellcolor[HTML]{58C765}{\textcolor{white}{6.93}}\\
\hline
>=\$6\_<\$8 & 31.76 & 7.28 & 55,843 & \cellcolor[HTML]{22A884}{\textcolor{white}{12.72}}\\
\hline
>=\$8\_<\$10 & 39.62 & 9.08 & 82,235 & \cellcolor[HTML]{25858E}{\textcolor{white}{18.73}}\\
\hline
>=\$10\_<\$12 & 45.59 & 10.44 & 164,264 & \cellcolor[HTML]{440154}{\textcolor{white}{37.41}}\\
\hline
>=\$12\_<\$14 & 54.39 & 12.46 & 6,219 & \cellcolor[HTML]{ABDC32}{\textcolor{white}{1.42}}\\
\hline
>=\$14\_<\$16 & 65.65 & 15.04 & 1,741 & \cellcolor[HTML]{BBDF27}{\textcolor{white}{0.40}}\\
\hline
>=\$16\_<\$18 & 74.95 & 17.17 & 6,941 & \cellcolor[HTML]{A9DB33}{\textcolor{white}{1.58}}\\
\hline
>=\$18\_<\$20 & 82.86 & 18.98 & 10,843 & \cellcolor[HTML]{9BD93C}{\textcolor{white}{2.47}}\\
\hline
>=\$20 & 143.54 & 32.89 & 10,066 & \cellcolor[HTML]{9DD93B}{\textcolor{white}{2.29}}\\
\hline
\multicolumn{5}{l}{\rule{0pt}{1em}Only persons reached through CBT/CVA/hybrid modalities are included}\\
\end{tabular}
\end{table}

\hypertarget{implementing-partners}{%
\subsubsection{3.4 Implementing partners}\label{implementing-partners}}

\includegraphics{q2_5ws_report_pdf_files/figure-latex/partner-cash-values-1.pdf}

\hypertarget{beneficiaries}{%
\subsection{4. Beneficiaries}\label{beneficiaries}}

\hypertarget{beneficiary-types}{%
\subsubsection{4.1 Beneficiary types}\label{beneficiary-types}}

In Q2 2022, 22.19\% of beneficiaries were from host or local
communities, in comparison to 53.31\% for round 1. 11.27\% of
beneficiaries in Q2 were IDPs, compared to 6.38\% for Q1.

\includegraphics{q2_5ws_report_pdf_files/figure-latex/unnamed-chunk-3-1.pdf}

\hypertarget{evidence-of-food-insecurity-status}{%
\subsubsection{4.2 Evidence of food insecurity
status}\label{evidence-of-food-insecurity-status}}

Of the food security activities reported, very few provided details
about the food insecurity status of beneficiaries. This makes it
difficult ot determine whether interventions are truly reaching those
most in need.

In general, the food insecurity status of the beneficiaries of
multi-purpose cash transfers were much better documented than the
statuses of those who received food distributions.

Mismatch between food insecurity status and activity (Q1 \& Q2 2022)

activity

food\_insecurity\_status

beneficiaries

\%of\_group

food distributions, moderate

Food secure

31,713

1.45

food distributions, moderate

Moderately food insecure

48,545

2.22

food distributions, moderate

Severely food insecure

206,922

9.48

food distributions, moderate

NA

1,895,565

86.84

food distributions, severe

Food secure

13,601

3.47

food distributions, severe

Moderately food insecure

13,644

3.48

food distributions, severe

Severely food insecure

67,956

17.34

food distributions, severe

NA

296,787

75.71

multi-purpose cash transfer, moderate

Food secure

16,183

45.67

multi-purpose cash transfer, moderate

Moderately food insecure

14,557

41.08

multi-purpose cash transfer, moderate

NA

4,695

13.25

multi-purpose cash transfer, severe

Food secure

41

0.06

multi-purpose cash transfer, severe

Severely food insecure

72,230

99.92

multi-purpose cash transfer, severe

NA

16

0.02

Evidence of beneficiaries' food insecurity status provided to the
cluster include:

\begin{table}

\caption{\label{tab:table-fs-evidence}Evidence of food insecurity status in Q1 & Q2 2022}
\centering
\begin{tabular}[t]{l|r|r}
\hline
evidence & beneficiaries & \%\_beneficiaries\\
\hline
No evidence & 2,757,212 & 90.90\\
\hline
Armed conflict & 147,088 & 4.85\\
\hline
community-based beneficiary selection & 34,407 & 1.13\\
\hline
Post-distribution monitoring & 31,347 & 1.03\\
\hline
Acceptable FCS & 29,071 & 0.96\\
\hline
Beneficiary list and distribution list & 19,279 & 0.64\\
\hline
Regular reporting & 8,984 & 0.30\\
\hline
assessment, meeting minutes, payment & 3,020 & 0.10\\
\hline
Village Profile & 1,683 & 0.06\\
\hline
Based on Vulnerable Score (Vulnerable Criteria) & 608 & 0.02\\
\hline
Food distribution certificate & 308 & 0.01\\
\hline
Food Security and Livelihood Baseline Survey & 119 & 0.00\\
\hline
Provision grants of women led micro business activities & 30 & 0.00\\
\hline
\end{tabular}
\end{table}

The general lack of evidence of evidence of beneficiaries' food
insecurity status makes it difficult to justify to affected communities
and donors that the Food Security Cluster is reaching the most in need.
This highlights the need to promote a shared understanding of the
response through the development of a common prioritisation tool for
food security partners.

\hypertarget{beneficiary-disaggregation}{%
\subsubsection{4.3 Beneficiary
disaggregation}\label{beneficiary-disaggregation}}

In this section, a test is applied to determine if the disaggregated
numbers of beneficiaries reach have been copied and pasted -- a somewhat
common practice that sullies the quality of the data. To do this, the
proportions of each disaggregation group by partner have been compared
to how close they were to the mean for the entire group. To explain: if
partner A reported that 40\% of beneficiaries in an activity were adult
females, this percentage was then compared to the average percentage of
adult females for all other activities reported by that partner. This
measure whether or not the same proportions were copied and pasted
throughout the 5W form.

It is extremely unlikely that these percentages would be similar across
activities as implementing partners worked in an average of 42
locations.

In the plot below, the closer a value is to 0\% on the x-axis, the more
likely it is that it was copied and pasted. It is estimated that 73\% of
beneficiary disaggregation values were copied and pasted.

\includegraphics{q2_5ws_report_pdf_files/figure-latex/disagg-histogram-1.pdf}

Once the copy-pasted values are removed, this is the breakdown of
beneficiaries by disaggregation group:

\includegraphics{q2_5ws_report_pdf_files/figure-latex/barplot-disagg-1.pdf}

\hypertarget{partners}{%
\subsection{5. Partners}\label{partners}}

\hypertarget{implementing-partner}{%
\subsubsection{5.1 Implementing partner}\label{implementing-partner}}

There are 55 partners that were involved in direct implementation that
have reported achievements in second quarter of 2022, in comparison with
44in the first quarter. These implementing partners corresponded to a
total of 26 reporting organisations. The largest reporting organisation,
2690, had 24 implementing partners.

Reporting organisations with the most implementing partners

report\_org\_code

implementing\_partners

org\_2690

24

org\_3536

7

org\_8415

7

org\_2625

4

org\_3422

4

org\_2214

3

org\_5369

3

All others had 1 or 2 implementing partners

The interactive plot below shows the number of beneficiaries and
townships reached by implementing partner.

13 partners (21\% of the total) have a presence in more than 5
townships. 8 partners are present in more than 10 townships.

\includegraphics{q2_5ws_report_pdf_files/figure-latex/plotly-partner-scatter-1.pdf}

\hypertarget{monthly-progress-by-partner}{%
\subsubsection{5.2 Monthly progress by
partner}\label{monthly-progress-by-partner}}

Organisations 6197, 2690 and 5722 have implemented the majority of their
activities in the second quarter of 2022. The thick grey line shows an
organisation's progress from last year.

\includegraphics{q2_5ws_report_pdf_files/figure-latex/partner-progress-facet-line-1.pdf}

The table below lists the top 15 partners by number of beneficiaries
reached in 2022.

\begin{table}

\caption{\label{tab:unnamed-chunk-5}Top implementing partners by beneficiaries reached in 2022 (Q1 & Q2)}
\centering
\begin{tabular}[t]{l|r|r|r|r|r}
\hline
org\_code & ben\_q1 & rank\_q1 & ben\_q2 & rank\_q2 & total\_ben\\
\hline
org\_8540 & 380,478 & 1 & 4,502 & 25 & 384,980\\
\hline
org\_5722 & 207,512 & 2 & 77,743 & 5 & 285,255\\
\hline
org\_9693 & 122,397 & 6 & 78,569 & 4 & 200,966\\
\hline
org\_4933 & 85,627 & 11 & 95,475 & 3 & 181,102\\
\hline
org\_5440 & 116,142 & 7 & 52,944 & 10 & 169,086\\
\hline
org\_6827 & 159,724 & 3 &  &  & 159,724\\
\hline
org\_1206 & 156,433 & 4 &  &  & 156,433\\
\hline
org\_2690 & 24,753 & 16 & 124,142 & 1 & 148,895\\
\hline
org\_9566 & 89,365 & 10 & 57,723 & 9 & 147,088\\
\hline
org\_3315 & 131,861 & 5 & 441 & 43 & 132,302\\
\hline
org\_5677 & 95,835 & 9 & 23,473 & 13 & 119,308\\
\hline
org\_6197 &  &  & 113,704 & 2 & 113,704\\
\hline
org\_8004 & 69,918 & 12 & 40,490 & 11 & 110,408\\
\hline
org\_6792 & 105,410 & 8 &  &  & 105,410\\
\hline
org\_6130 & 35,385 & 15 & 61,280 & 8 & 96,665\\
\hline
\end{tabular}
\end{table}

\hypertarget{donors}{%
\subsubsection{5.3 Donors}\label{donors}}

The table below summarises the reach and scope (in terms of geographic
extent and number of organisations supported) of donors who support at
least two reporting organisations. FCDO and LIFT support the most
expansive programmes.

\begin{table}

\caption{\label{tab:unnamed-chunk-6}Organisations supported and geographic reach by donor}
\centering
\begin{tabular}[t]{l|r|r|r|r}
\hline
donor & report\_orgs & implementing\_orgs & states & townships\\
\hline
FCDO & 6 & 8 & 6 & 19\\
\hline
LIFT & 6 & 10 & 6 & 15\\
\hline
MHF & 5 & 6 & 4 & 7\\
\hline
ECHO & 3 & 4 & 3 & 5\\
\hline
BHA & 2 & 5 & 4 & 11\\
\hline
GIZ & 2 & 2 & 2 & 7\\
\hline
org\_5677 & 2 & 2 & 6 & 15\\
\hline
org\_6793 & 2 & 3 & 3 & 9\\
\hline
org\_7904 & 2 & 3 & 2 & 7\\
\hline
org\_8415 & 2 & 2 & 3 & 15\\
\hline
\multicolumn{5}{l}{\rule{0pt}{1em}Only showing donors supporting more than one reporting partner.}\\
\end{tabular}
\end{table}

Sagaing, Shan (East) and Ayeyarwady have the fewest number of donors
present.

\begin{table}

\caption{\label{tab:table-donor-states}Number of donors by state}
\centering
\begin{tabular}[t]{l|r|r}
\hline
state & donors & implementing\_partners\\
\hline
Kayin & 14 & 13\\
\hline
Kachin & 13 & 14\\
\hline
Rakhine & 11 & 18\\
\hline
Kayah & 10 & 5\\
\hline
Mon & 7 & 6\\
\hline
Shan (South) & 7 & 8\\
\hline
Mandalay & 5 & 3\\
\hline
Shan (North) & 5 & 8\\
\hline
Bago (East) & 4 & 3\\
\hline
Chin & 4 & 4\\
\hline
Yangon & 4 & 14\\
\hline
Magway & 3 & 3\\
\hline
Tanintharyi & 3 & 3\\
\hline
Ayeyarwady & 2 & 4\\
\hline
Sagaing & 2 & 3\\
\hline
Shan (East) & 2 & 2\\
\hline
\end{tabular}
\end{table}

However, as shown by the table below, even though the majority of
partners reported their donors, the omission of data from three key
partners has resulted in the vast majority of reported beneficiaries not
being associated with any donor.

\begin{table}

\caption{\label{tab:table-donors-beneficiaries}Top donors by beneficiaries reached}
\centering
\begin{tabular}[t]{l|r|r}
\hline
donor & beneficiaries & \%\_beneficiaries\\
\hline
No donor specified & 2,397,314 & 79.04\\
\hline
org\_6793 & 156,303 & 5.15\\
\hline
FCDO & 94,653 & 3.12\\
\hline
CERF & 71,253 & 2.35\\
\hline
AICS & 63,986 & 2.11\\
\hline
org\_5677 & 49,279 & 1.62\\
\hline
BPRM & 31,640 & 1.04\\
\hline
org\_7904 & 20,213 & 0.67\\
\hline
DFAT & 17,329 & 0.57\\
\hline
BHA & 16,356 & 0.54\\
\hline
org\_7002 & 13,851 & 0.46\\
\hline
AAP & 12,392 & 0.41\\
\hline
MHF & 12,303 & 0.41\\
\hline
NZMFAT & 10,791 & 0.36\\
\hline
ECHO & 7,854 & 0.26\\
\hline
\end{tabular}
\end{table}

Below is a table of beneficiaries who are missing donors, grouped by
state.

\begin{table}

\caption{\label{tab:missing-donor}Reported beneficiaries with missing donor data}
\centering
\begin{tabular}[t]{l|r|r}
\hline
state & beneficiaries & partners\\
\hline
Yangon & 1,752,158 & 9\\
\hline
Rakhine & 421,076 & 9\\
\hline
Kayah & 98,394 & 1\\
\hline
Kachin & 69,031 & 3\\
\hline
Shan (North) & 18,420 & 5\\
\hline
Chin & 16,389 & 2\\
\hline
Shan (South) & 9,416 & 2\\
\hline
Sagaing & 7,650 & 2\\
\hline
Kayin & 4,505 & 2\\
\hline
Shan (East) & 275 & 1\\
\hline
\end{tabular}
\end{table}

\hypertarget{comparison-with-targets}{%
\subsection{6. Comparison with targets}\label{comparison-with-targets}}

\hypertarget{reached-vs-target-by-township}{%
\subsubsection{6.2 Reached vs target by
township}\label{reached-vs-target-by-township}}

The specifics of each township can be reviewed with the interactive plot
below. Each point is a township, with the size indicating the number of
beneficiaries. The x-axis indicates the target population by township
and the y-axis shows the number of beneficiaries reached in Q1 and Q2
2022.

The red line down the middle represents reaching 100\% of the target.
Townships above this line have reached more beneficiaries than their
target and townships below the line have not met their target yet. The
further away a township is from the red line, the further above or below
its target it is. Mouse over each of the townships to see more details.

The 13 townships along the extreme left side of the plot have
beneficiaries but do not have targets (their targets have just been
coded as 1 so that they show up on the plot). 209 townships with targets
have not been reached.

\includegraphics{q2_5ws_report_pdf_files/figure-latex/plotly-tsp-comparison-reached-target-1.pdf}

\hypertarget{map-of-beneficiaries-reached-in-q1-q2-2022-vs-target}{%
\subsubsection{6.2 Map of beneficiaries reached in Q1 \& Q2 2022 vs
target}\label{map-of-beneficiaries-reached-in-q1-q2-2022-vs-target}}

\includegraphics{q2_5ws_report_pdf_files/figure-latex/maps-ben-target-1.pdf}

\hypertarget{interactive-reference-table}{%
\subsubsection{6.3 Interactive reference
table}\label{interactive-reference-table}}

There was an overallocation of resources in these relatively few areas
in 2021 and this has continued in the first quarter of 2022. In the
interactive table below, is a list of townships sorted by the gap
between the targeted population and beneficiaries reached in 2022. Any
of the columns can be sort; the search bars above each column can also
assist in filtering.

\includegraphics{q2_5ws_report_pdf_files/figure-latex/unnamed-chunk-7-1.pdf}

\end{document}
